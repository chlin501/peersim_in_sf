\documentclass[a4paper,11pt]{article}

\usepackage{a4wide}
\usepackage{graphicx}
\usepackage{hyperref}

\title{A Basic Event Driven Example for PeerSim 1.0}
\author{M\'ark Jelasity}

\begin{document}

\maketitle

\section{Introduction}

This document walks you through a basic example application of PeerSim,
that uses the \emph{event based} simulation model.
We assume that you are familiar with the examples described in
the short tutorials on the cycle based model: the basic working
of PeerSim, its configuration file, and the gossip-based averaging example,
where nodes collectively calculate the average of some parameter via
periodically exchanging messages and performing pairwise averaging steps.
The example discussed here will be the same: gossip-based averaging.
Only this time using the event based model, where message sending is modelled
in more detail.
This will allow us to observe problems with the protocol that remained hidden
in the cycle based simulations.

In the event based model, everything works exactly the same way as in the
cycle based model, except time management and the way control is passed to
the protocols.
\emph{Protocols} that are not executable (that are used only to store
data, such as some linkable protocols that only store neighbors, or vectors that
store numeric values) can be applied and initialized in exactly the same way.
\emph{Controls} in any package outside package \emph{peersim.cdsim} can
be used as well. By default controls are called in each cycle in the cycle
based model. In the event based model they have to be scheduled explicitly,
since there are no cycles.

Evidently, we can write controls that are specific to the event based model,
that is, that are able to send events (messages) to protocols.
In many cases, this will be necessary because often the system will be
driven completely or partially by external events such as queries by users,
that can best be
modeled by controls that generate these events and thereby drive the
execution of the simulation.

There are components that are not available.
This includes any component that relies on the static class
\emph{peersim.cdsim.CDState} which offers an interface to read cycle-specific
global state.
Our experience is that many components written for the cycle based model
that rely on this state can be easily modified to remove this dependency.

However, maybe a bit surprisingly, protocols that implement the cycle based
interface \emph{peersim.cdsim.CDProtocol} can be utilized in the event
based model as well; we will explain later how.
But it should be noted immediately that this must be done with care, because
in most of the cases it does not make any sense.
However, this feature does have its very useful applications: it makes it
possible to easily invoke protocols periodically, a feature that is
characteristic to practically all P2P protocols in connection with
housekeeping, failure detection and sending heartbeat messages.

\section{Event based averaging: the protocol}

Let us start with the Java class that implement our averaging protocol
in the event based model, that begins like this:
\footnotesize
\begin{verbatim}
package example.edaggregation;

import peersim.vector.SingleValueHolder;
import peersim.config.*;
import peersim.core.*;
import peersim.transport.Transport;
import peersim.cdsim.CDProtocol;
import peersim.edsim.EDProtocol;

/**
* Event driven version of epidemic averaging.
*/
public class AverageED extends SingleValueHolder
implements CDProtocol, EDProtocol {
\end{verbatim}
\normalsize

The first thing we notice is that we implement interface \emph{EDProtocol}
and also \emph{CDProtocol}.
The former will allow the class to process incoming messages.
The latter is a bit confusing, as it belongs to the cycle based model.
First of all, protocols need not implement this interface in the event
based model.
However, protocols that would like to get control periodically can achieve
this effect using this trick of implementing \emph{CDProtocol}, and
setting a \emph{CDScheduler} in the configuration, as we will describe
below.
This has several advantages over fiddling with timers and so on.
The code is much cleaner, the periodic execution that is managed by a separate
component is modular itself and can be configured separately, and finally,
we can port our old cycle based protocols much easier to the event based
model.

\footnotesize
\begin{verbatim}
/**
 * @param prefix string prefix for config properties
 */
public AverageED(String prefix) { super(prefix); }
\end{verbatim}
\normalsize

Our simple protocol does not read any configuration parameters.
Now let us turn to the implementation of the cycle based interface.
This method defines the activity to be performed periodically.

\footnotesize
\begin{verbatim}
/**
 * This is the standard method the define periodic activity.
 * The frequency of execution of this method is defined by a
 * {@link peersim.edsim.CDScheduler} component in the configuration.
 */
public void nextCycle( Node node, int pid )
{
        Linkable linkable = 
                (Linkable) node.getProtocol( FastConfig.getLinkable(pid) );
        if (linkable.degree() > 0)
        {
                Node peern = linkable.getNeighbor(
                                CommonState.r.nextInt(linkable.degree()));
                
                // XXX quick and dirty handling of failures
                // (message would be lost anyway, we save time)
                if(!peern.isUp()) return;
                
                AverageED peer = (AverageED) peern.getProtocol(pid);
                
                ((Transport)node.getProtocol(FastConfig.getTransport(pid))).
                        send(
                                node,
                                peern,
                                new AverageMessage(value,node),
                                pid);
        }
}
\end{verbatim}
\normalsize

What we need to observe here is the stuff specific to the event based
model.
This boils down to handling the transport layer.
First of all, class \emph{FastConfig} gives us a way to access the transport
layer that was configured for this protocol.
Using this transport layer, we can send messages to protocols on other nodes.
A message can be an arbitrary object.
Since the simulator is not distributed, there is no trouble with
serialization, etc; the object will be stored by reference.

The target protocol is defined by the target node \texttt{peern}, and
the protocol identifier of the target protocol \texttt{pid}.
In our case, we send a message to the same protocol on a different node.
Evidently, the target protocol has to implement the \emph{EDProtocol}
interface.

\footnotesize
\begin{verbatim}
/**
* This is the standard method to define to process incoming messages.
*/
public void processEvent( Node node, int pid, Object event ) {
                
        AverageMessage aem = (AverageMessage)event;
        
        if( aem.sender!=null )
                ((Transport)node.getProtocol(FastConfig.getTransport(pid))).
                        send(
                                node,
                                aem.sender,
                                new AverageMessage(value,null),
                                pid);
                                
        value = (value + aem.value) / 2;
}

}
\end{verbatim}
\normalsize

The method above is specified in \emph{EDSimulator} and is supposed to
handle icoming messages.
In our example, there is only one type of message.
All we need to check whether the sender is null, because if it is, it
means that we do not need to answer the message (it is already an answer).
If we need to answer, we do this the same way as we have seen already, through
the transport layer.

\footnotesize
\begin{verbatim}
/**
* The type of a message. It contains a value of type double and the
* sender node of type {@link peersim.core.Node}.
*/
class AverageMessage {

        final double value;
        /** If not null,
        this has to be answered, otherwise this is the answer. */
        final Node sender;
        public AverageMessage( double value, Node sender )
        {
                this.value = value;
                this.sender = sender;
        }
}

\end{verbatim}
\normalsize

This private class is the type of the message that the protocol uses.
It is private because no other component has anything to do with the
message type.

\section{Event based averaging: the configuration}

Let us examine a configuration file that can be used to run the event based
simulation.
It is very similar to the cycle based configuration, the difference is
slight, but important.

\footnotesize
\begin{verbatim}
# network size
SIZE 1000

# parameters of periodic execution
CYCLES 100
CYCLE SIZE*10000

# parameters of message transfer
# delay values here are relative to cycle length, in percentage,
# eg 50 means half the cycle length, 200 twice the cycle length, etc.
MINDELAY 0
MAXDELAY 0
# drop is a probability, 0<=DROP<=1
DROP 0
\end{verbatim}
\normalsize

We have just defined a number of constants to make the configuration file
cleaner and easier to change from the command line.
For example, \texttt{CYCLE} defines the length of a cycle.

\footnotesize
\begin{verbatim}
random.seed 1234567890
network.size SIZE
simulation.endtime CYCLE*CYCLES
simulation.logtime CYCLE
\end{verbatim}
\normalsize

Parameter \texttt{simulation.endtime} is the key thing here: it tells the
simulator when to stop.
The internal representation of time is long (64 bit integer).
It is zero at startup time and it is advanced by message delays.
Simulation stops when the event queue is empty (nothing left to do) or
if all the events in the queue are scheduled for a time later than the
specified end time.

The simulator prints indications on the standard error about the
progress of time.
Parameter \texttt{simulation.logtime} specifies the frequency of these
messages.

\footnotesize
\begin{verbatim}
################### protocols ===========================

protocol.link peersim.core.IdleProtocol

protocol.avg example.edaggregation.AverageED
protocol.avg.linkable link
protocol.avg.step CYCLE
protocol.avg.transport tr

protocol.urt UniformRandomTransport
protocol.urt.mindelay (CYCLE*MINDELAY)/100
protocol.urt.maxdelay (CYCLE*MAXDELAY)/100

protocol.tr UnreliableTransport
protocol.tr.transport urt
protocol.tr.drop DROP
\end{verbatim}
\normalsize

Here we configure our protocol (\texttt{avg}) and specify the
overlay network (\texttt{link}) and the transport layer (\texttt{tr}).
We also have to specify the \texttt{step} scheduling parameter, familiar
from the cycle based model.
This is because we have implemented the cycle based interface, so we need
to specify how long a cycle is, in order to be able to make use of it.

The overlay network is just a container of links that will remain constant
throughout the simulation and that will be initialized as shown below.

The transport layer is also configured as a protocol. It models both random
delays and message drops.
First we define a transport layer that has random delay (\texttt{urt}) and
then we wrap it in a generic wrapper that takes any transport layer, and
drops messages with a given probability (\texttt{tr}).

\footnotesize
\begin{verbatim}
################### initialization ======================

init.rndlink WireKOut
init.rndlink.k 20
init.rndlink.protocol link

init.vals LinearDistribution
init.vals.protocol avg
init.vals.max SIZE
init.vals.min 1

init.sch CDScheduler
init.sch.protocol avg
init.sch.randstart
\end{verbatim}
\normalsize

Here the only component that is specific to the event based model is
\texttt{sch}.
It is responsible for scheduling the periodic call of the cycle based
interface (\texttt{nextCycle}).
In this configuration, this component will first assign a random point in
time between 0 and \texttt{CYCLE} to all nodes, which will be the first
time \texttt{nextCycle} is called on protocol \texttt{avg}.
Then the next calls will happen in intervals of exactly \texttt{CYCLE} time
steps regularly.
More advanced methods also exist, besides, the scheduling can be customized
by class extension; we do not go into the details here.

\footnotesize
\begin{verbatim}
################ control ==============================

control.0 SingleValueObserver
control.0.protocol avg
control.0.step CYCLE
\end{verbatim}
\normalsize

This we have seen already.
Note that we need to specify the \texttt{step} parameter here as well,
just like for protocol \texttt{avg}.
This will specify how often this control will be called.
Otherwise controls can be scheduled the same way as in the cycle based model,
only there is no default \texttt{step}, because there are no cycles.


\section{Let's play}

\section{Disclaimer}

This simple example is only to get you started.
You have not seen all.
It is highly recommended to study the class documentation of the relevant
packages (\emph{peersim.edsim, peersim.transport}) for maximal control,
and it does not hurt if you are also familiar with the generic components
of PeerSim as well.

\end{document}


